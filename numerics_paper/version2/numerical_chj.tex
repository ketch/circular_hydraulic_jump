\documentclass[preprint, 11pt]{article}

\usepackage[T1]{fontenc}
\usepackage[utf8]{inputenc}
\usepackage[hmargin=0.9in, vmargin=1.25in]{geometry}
\usepackage[babel,german=quotes]{csquotes}
\usepackage{subfig}
\usepackage{graphicx}
\usepackage[colorlinks=true,citecolor=blue,urlcolor=blue]{hyperref}
\usepackage{caption}
\usepackage{amsmath}          % writing mathematical formulas
\usepackage{amsthm}           % writing mathematical theorems 
\usepackage{amssymb}          % writing mathematical symbols
\usepackage{bm, amsfonts}     % writing bold mathematical symbols
\usepackage{xcolor}
\usepackage{fixmath}
\usepackage{tikz}
\usepackage{bm}
\usepackage{makecell}
\usepackage{mathdots}
\usepackage{algpseudocode}
\usepackage{algorithm}

\newcommand{\W}{{\mathcal W}}
\newcommand{\A}{{\mathcal A}}
\newcommand{\bff}{{\bf f}}
\newcommand{\bfu}{{\bf u}}
\newcommand{\bfv}{{\bf v}}
\newcommand{\bfq}{{\bf q}}
\newcommand{\bfx}{{\bf x}}

\title{Numerical simulation and entropy dissipative cure of the 
  carbuncle instability for the shallow water circular hydraulic jump}

\author{
    David I. Ketcheson \and
    Manuel Quezada de Luna
}
\begin{document}

\maketitle

\begin{abstract}

\end{abstract}


\section{Introduction}

\subsection{The circular hydraulic jump}

\begin{itemize}
  \item Describe the formation of the circular hydraulic jump, some properties, etc. 
  \item Close this section stating that for some flow regimes, the CHJ becmes unstable and its 
    shape deviates from a circular shape. For high viscosity flows polygonal shapes are observed 
    and for low viscosity flows the shape appears to be chaotic. 
  \item We can use section 1.1 in the original version of the paper.
\end{itemize}

\subsection{Numerical shock instabilities}

\begin{itemize}
  \item Describe the carbuncle instability starting with Euler and going to SWEs. 
  \item Mention that we observe the cc instability for the CHJ as well. 
  \item Describe that in this case, the situation is more complicated than in other problems since we 
    expect the correct solution to have physical instabilities triggered by round off error. 
  \item Based on existing literature we know that we can use highly dissipative and robust methods to 
    avoid the formation of carbuncle instabilities. However, in this case, these methods also tend to 
    dissipate other important features of the solution unless the mesh is highly refined. 
  \item We are interested in methods that prevent the formation of cc instability in the CHJ 
    but preserve other features and physical instabilities. 
  \item There are multiple ideas and methods proposed to fix the cc insitability for the Euler 
    and recently for the SWEs. 
  \item It is believed that the cc instability occurs due to lack or reduced viscosity in the shear contact waves. 
  \item Some methods also propose to improve the entropy stability properties of the method. In particular, to induce or guarantee that the entropy is dissipated at shocks. 
  \item Based on some of these ideas we start with Rusanov's method using the cell averages of the solution. 
    This is a first-order method that is entropy stable and positivity preserving. We enhance its robustness by 
    overestimating the wave speed when strong shear is present in the solution. The result is a robust solver 
    that do not develop cc instabilities. However, the method is highly dissipative and, as a result, 
    many other important features of the solution are dissipated. 
  \item We then blend this highly dissipative method with a FV method based on Roe's average. 
  \item After this we introduce the second order corrections from Law-Wendroff's method. 
\end{itemize}

\subsection{Our contribution}
\begin{itemize}
\item We present a new problem where the cc instability occurs. In our opinion, this is a more challenging 
  problem than the standard bow problem since we expect the solution to be non-steady. 
  We aim to preserve instabilities that are inherently present in the CHJ for certain regimes. And, 
  at the same time, we aim to prevent the cc instability. 
\item We explore the use of entropy dissipation and stability as cure of the cc instability for the SWEs. 
\item We propose specific methods to blend the highly dissipative and robust Rusanov's RP and the 
  accurate 3-wave Roe's solver for the SWEs. In particular we propose methods that induce dissipation of entropy
  near the shocks and other methods that guarantee entropy stability. 
\item We demonstrate that these schemes prevent the formation of cc instabilities not only 
  in the well known bow shock problem, but also in the CHJ while preserving other important features of 
  the solution. 
\end{itemize}


\section{The shallow water circular hydraulic jump}
The shallow water equations in two (horizontal) dimensions are
\begin{subequations} \label{eq:sw}
\begin{align}
    h_t + (hu)_x + (hv)_y & = 0 \\
    (hu)_t + \left(hu^2 + \frac{1}{2}gh^2\right)_x + (huv)_y & = 0 \\
    (hv)_t + (huv)_x + \left(hv^2 + \frac{1}{2}gh^2\right)_y & = 0.
\end{align}
\end{subequations}
These can be written in vector form as
\begin{align}
    q_t + f(q)_x + g(q)_y & = 0.
\end{align}

\subsection{Semi-analytical steady solution under rotational symmetry}

By assuming rotational symmetry in \eqref{eq:sw}, one obtains the
system (in one spatial variable)
\begin{subequations} \label{eq:rsw}
\begin{align}
    (rh)_t + (rhu)_r & = 0 \label{mass1} \\
    (rhu)_t + (rhu^2)_r + r \left(\frac{1}{2}gh^2\right)_r = 0. \label{mom1}
\end{align}
\end{subequations}
Steady-state solutions of \eqref{eq:rsw} satisfy
\begin{subequations}
\begin{align}
    rhu & = C \\
    h'(r) & = \frac{h}{\frac{g}{\beta^2} r^3 h^3 -r} = \frac{h}{r} \cdot \frac{F^2}{1-F^2} \label{eq:dh}
\end{align}
\end{subequations}
for some $C$ independent of $r$.  Here $F=|u|/\sqrt{gh}$ is the Froude number.
We see that two types of steady profiles exist, depending on whether the flow
is subcritical ($|F|<1$) or supercritical ($|F|>1$).  No smooth steady solution can
include both regimes, since the right hand side of \eqref{eq:dh} blows up when $F=1$.

The steady, rotationally-symmetric circular hydraulic jump involves supercritical
flow for $r<r_0$ and subcritical flow for $r>r_0$, where $r_0$ is the jump radius.
The jump itself takes the form of a stationary shock wave.  The Rankine-Hugoniot jump
conditions specify that for such a shock,
\begin{align} \label{eq:RH}
    h_+ - h_- & = \frac{-3h_- + \sqrt{h_-^2 + 8 h_- u_-^2/g}}{2} = \frac{3h_-}{2}\left(\sqrt{1+\frac{8}{9}(F_-^2-1)}-1\right),
\end{align}
where the subscripts $+, -$ denote states just inside or outside the jump radius, respectively.

A steady-state, rotationally symmetric solution can be given for an annular region as follows:

\begin{enumerate}
    \item Specify the depth and velocity at the inner boundary (near the jet) and outer boundary.
    \item Integrate \eqref{eq:dh} from both boundaries.
    \item Find a radius $r_0$ where the matching condition \eqref{eq:RH} is satisfied.
\end{enumerate}
Due to the nature of solutions of \eqref{eq:dh}, it can be shown that the required jump
radius $r_0$ always exists if the prescribed flow is supercritical at the inner boundary
and subcritical at the outer boundary.

Numerical tests suggest that the solution obtained in this way is a stable equilibrium
of the reduced one-dimensional shallow water equations \eqref{eq:rsw}.  This
solution provides a useful starting point for testing the stability of the CHJ
as a solution of the full two-dimensional shallow water equations \eqref{eq:sw}.

\subsection{Formation of the CHJ}

\section{Numerical methods}
In this work, we consider Riemann-based finite volume methods.
Considering a one dimensional hyperbolic conservation law,
the cell average of the numerical solution given by a finite volume scheme (with forward Euler) 
can be written as follows:
\begin{align}\label{first-order_FV}
  u_i^{n+1}=u_i^n-\frac{\Delta t}{\Delta x}\left[F_{i+1/2}-F_{i-1/2}\right],
\end{align}
where $F_{i+1/2}=F_{i+1/2}(u^n)$ is a numerical flux; i.e., a discretization of the flux. 
This numerical flux is usually is composed of two parts, a consistent term that approximates 
the physical flux at the interface between cells $i$ and $i+1$, and a stabilization term. 
For instance, the well known Rusanov's numerical flux is given by 
\begin{align}\label{rusanov_num_flux}
  F_{i+1/2}(z)=\frac{\bff(z_{i+1})+\bff(z_i)}{2} - \lambda_{i+1/2}(z_{i+1}-z_i),
\end{align}
where $\lambda_{i+1/2}$ is an upper bound on the wave speed associated with the Riemann problem 
composed of the states $u_i$ and $u_{i+1}$. 
If $z=u$ in \eqref{rusanov_num_flux}
%; i.e., if the Rusanov's numerical flux is obtained considering 
%the cell averages of the solution 
and if the time step is small enough, 
then the method \eqref{first-order_FV} is positivity preserving and entropy stable, see for instance XX. 
However, the accuracy of the numerical flux, and hence of the method, would be 
limited to first-order. To improve the accuracy properties of Rusanov's method, 
two alternatives are common. 
The numerical flux $F_{i+1/2}$ can be approximated based on polynomial 
reconstructions around the interface $i+1/2$. 
For smooth solutions, this strategy leads to arbitrarily high-order
methods. Alternatively, second-order accuracy can be obtained by removing some of the 
numerical dissipation introduced by the stabilization component of the numerical flux. 
The second-order Lax-Wendroff method XX, which we use in this work, 
is a popular example of this approach. 
Both high-order corrections to Rusanov's method require careful treatment to guarantee 
the solution is still positivity preserving and/or entropy stable. 

Our main objective is to study the instabilities in the CHJ and to propose a Riemann solver
that is carbunclue free but do not dissipate important features of the solution. 
To do that we combine the dissipative Rusanov's numerical flux with a Roe's average based
approximate Riemann solver. A similar idea, to blend Rusanov's and a Roe's average 
Riemann solver has been explored before in XX to solve the Euler equations. The blending 
function in the aformentioned work is based on XX....

After introducing the blended Riemann solver, we use it with the first order method 
\eqref{first-order_FV} and then as building block of the Lax-Wendroff method. 

\subsection{Lax-Wendroff-LeVeque method}

\section{Riemann solvers}

\subsection{Two- versus three-wave solvers}
The general finite volume method \eqref{first-order_FV} can be written in terms of left- 
and right-going waves emanating from the interfaces XX. The net contribution of these waves 
is called fluctuations. Based on fluctuations, method \eqref{first-order_FV} can be written as 
\begin{align}\label{FV_via_fluct}
  u_i^{n+1}=u_i^n-\frac{\Delta t}{\Delta x}\left[\A^+\Delta Q_{i-1/2}+\A^-\Delta Q_{i+1/2}\right],
\end{align}
where $\A^+\Delta Q_{i-1/2}$ and $\A^-\Delta Q_{i+1/2}$ are the right- and left-going fluctuations 
emanating from the interfaces $i-1/2$ and $i+1/2$, respectively.
The waves and their speed of propagation (to compute the fluctuations) are tightly related to 
Riemann solvers. For the shallow water equations in two-dimensions, the structure of the exact (and some 
approximated) Riemann solvers contains three waves. This is the case, for example, of the well known 
Roe's Riemann solver, see for instance XX. Simpler approaches are also possible. For instance, 
method XX with the Rusanov's numerical flux XX can be rewritten in the form XX by considering 
only two waves that propagate with the same speed in opposite directions.


%\clearpage
%We consider the implementation of the Lax-Wendroff method in Clawpack. 
%Following XX, and assuming a one-dimensional problem, the method can be written as follows:
%\begin{align*}
%  U_i^{n+1}=U_i^n
%  -\frac{\Delta t}{\Delta x}\left[\A^+\Delta Q_{i-1/2}+\A^-\Delta Q_{i+1/2}\right]
%  -\frac{\Delta t}{\Delta x}\left[\tilde F_{i+1/2}-\tilde F_{i-1/2}\right],
%\end{align*}
%where XX and XX are right- and left-going fluctuations emanating from the interfaces XX and XX, respectively. 
%The terms XX and XX are flux corrections that improve the accuracy to second order. 
%. In this section we focus on the first order method; i.e., we consider 
%XX 
%
%and propose different Riemann solvers. Once the Riemann solvers are defined for the low-order method, 
%the second order corrections are incorporated following the same steps for all cases. Therefore, we focus our 
%attention to design of the Riemann solvers. 

%\subsection{Roe's Riemann solver}
\begin{align*}
  F_{i+1/2}=\frac{\bff(u_{i+1})+\bff(u_i)}{2}+\frac{1}{2}|\hat A_{i+1/2}|(u_{i+1}-u_i)
\end{align*}

%\subsection{Rusanov's Riemann solver}

\clearpage
\section{Entropy dissipative Riemann solver}
Given a hyperbolic system of conservation laws 
\begin{align}\label{cons_law}
  \frac{\partial \bfu(\bfx)}{\partial t} + \nabla\cdot \bff(\bfu(\bfx))=0, \quad \bfx\in\Omega,
\end{align}
a convex function $\eta(\bfu)$ is called an entropy if there exists a function $\bfq(\bfu)$, called 
entropy flux, such that XX. A weak solution of XX is called an entropy solution if 
\begin{align}
  \frac{\partial \eta(\bfu(\bfx))}{\partial t} + \nabla\cdot \bfq(\bfu(\bfx))\leq 0, \quad \bfx\in\Omega,
\end{align}
holds, in a distribution sense, for any entropy pair $(\eta(\bfu),\bfq(\bfu))$ of \eqref{cons_law}.
It is desirable for a numerical method to satisfy a discrete version of the entropy inequality XX. 
This is, however, not true in general. 
In general, an entropy stable scheme introduces some type of nonlinear stabilization, which is 
particularly important for solutions with strong discontinuities that might converge to non-physical 
weak solutions. 
In this work, we propose an entropy stable approximate Riemann solver for the one dimensional 
shallow water equations. We use this Riemann solver as building block within the Lax-Wendroff 
method, as presented in XX and implemented in Clawpack XX.

\subsection{Entropy residual}

Let $S_i$ be the set of cells containing cell $i$ and those that share a face with it. 
Based on XX, we consider 
\begin{align}\label{ent_residual}
  \int_{S_i} \bfv(\bfu) \cdot \left[ \frac{\partial \bfu}{\partial t} + \nabla\cdot \bff(\bfu)\right]d\bfx
  =\int_{S_i} \left[\frac{\partial \eta(\bfu)}{\partial t} + \bfv(\bfu) \cdot \nabla\cdot \bff(\bfu)\right] d\bfx
\end{align}
as a measurement of the entropy production in $S_i$. To avoid approximating the time derivative of the entropy, 
we follow XX, XX and use XX to approximate XX=XX, which holds in smooth regions. Then we define 
\begin{align}\label{Ri}
  R_i := \frac{N_i}{D_i}, 
\end{align}
with
\begin{align*}
  N_i=
  \left|\int_{S_i} \left[\bfv(\bfu) \cdot \nabla\cdot \bff(\bfu) - \nabla\cdot\bfq(\bfu) \right] d\bfx\right|,
\end{align*}
and $D_i$ being an upper bound of $N_i$ that acts as normalization so that $0\leq R_i\leq 1$.
Note that $N_i=0$ if $\bfu$ is smooth in $S_i$. 
In our implementation, we use
\begin{align*}
  N_i\approx \left|\bfv(\bfu_i)\cdot \int_{S_i}\nabla\cdot \bff(\bfu)d\bfx 
  -\int_{S_i}\nabla\cdot\bfq(\bfu) d\bfx\right|,
  \quad 
  D_i = ||\bfv(\bfu_i)||_{\ell^2}\left|\left|\int_{S_i}\nabla\cdot\bff(\bfu)d\bfx\right|\right|_{\ell^2}
  +\left|\int_{S_i}\nabla\cdot\bfq(\bfu)d\bfx\right|.
\end{align*}
In the next section we use $R_i$ to blend the Rusanov's and the Roe's Riemann solvers. 

\subsection{Blended Riemann solver}
In this section, we use \eqref{Ri}, which is a normalized measurement of the entropy residual \eqref{ent_residual}, 
to blend the Rusanov's and the Roe's Riemann solvers. To do this, we define the following numerical flux:
\begin{align}\label{num_flux_ev}
  F_{i+1/2} = \frac{\bff(\bfu_{i+1})+\bff(\bfu_i)}{2} 
  - \frac{1}{2} \left( R_{i+1/2}\lambda_{i+1/2}^{\max} + (1-R_{i+1/2})|\hat A_{i+1/2}| \right)(\bfu_{i+1}-\bfu_{i}),
\end{align}
and similarly for $F_{i-1/2}$. Here $R_{i+1/2}=\max\{R_{i+1},R_i\}$, $\lambda_{i+1/2}^{\max}$ is the 
upper bound on the wave speed used in the Rusanov's Riemann solver, see \S XX, and $\hat A_{i+1/2}$
is the Roe's averaged flux Jacobian, see \S XX. 

%In terms of fluctuations, 
The first-order method is given simply by \eqref{first-order_FV} 
with the numerical flux $F_{i+1/2}$ given by \eqref{num_flux_ev}. 
In smooth regions $R_{i+1/2}\approx 0$ so the numerical flux is essentially The Roe's numerical flux. 
Near discontinuities or steep gradients, $R_{i+1/2}$ becomes large. If, in particular, $R_{i+1/2}\approx 1$, 
then the numerical flux is essentially the Rusanov's numerical flux.  
%
The first-order method can also be written in terms of fluctuations. 
To do this we consider the waves ($\W^p_{i\pm 1/2}$, with $p=1,2,3$) 
from the Roe's solver and define 
\begin{align}
  \lambda_{i+1/2}^p := R_{i+1/2}\lambda_{i+1/2}^{\max} + (1-R_{i+1/2})|\hat \lambda_{i+1/2}^p|,
\end{align}
where $\hat\lambda_{i+1/2}^p$ is the $p$-th eigenvalue of the Roe's solver corresponding to the 
interface $i+1/2$. The fluctuations are then given by 
\begin{subequations}\label{ev_fluctuations}
\begin{align}
  \A^+\Delta Q_{i-1/2}&=\frac{1}{2}\sum_p \left(\hat\lambda_{i-1/2}^p + \lambda_{i-1/2}^p\right)\W_{i-1/2}^p, \\
  \A^-\Delta Q_{i+1/2}&=\frac{1}{2}\sum_p \left(\hat\lambda_{i+1/2}^p - \lambda_{i+1/2}^p\right)\W_{i+1/2}^p.
\end{align}
\end{subequations}

To implement the second-order Lax-Wendroff type method in Clawpack, the Riemann solver must provide 
the fluctuations, the waves and their speed. The fluctuations are simply \eqref{ev_fluctuations} 
and the waves are the Roe's waves $\W_{i+1/2}^p$, with $p=1,2,3$. 
In the framework of LeVeque, for a given interface $i+1/2$, 
we would need to consider a wave speed $s_{i+1/2}^p$ so that 
\begin{align*}
  \hat \lambda_{i+1/2}^p+\lambda_{i+1/2}^p = s_{i+1/2}^p+|s_{i+1/2}^p|, \qquad 
  \hat \lambda_{i+1/2}^p-\lambda_{i+1/2}^p = s_{i+1/2}^p-|s_{i+1/2}^p|,
\end{align*}
which is not possible to obtain in general (since the problem is overdetermined). Instead, we consider 
simply $s_{i+1/2}^p=\hat\lambda_{i+1/2}^p$ and study the consequences of this choice 
once the second-order corrections are applied.
The second-order Lax-Wendroff-LeVeque method for the one-dimensional problem is given by 
\begin{align}\label{second-order}
  u_i^{n+1}=u_i^n
  -\frac{\Delta t}{\Delta x}\left[\A^-\Delta Q_{i+1/2}+\A^+\Delta Q_{i-1/2}\right]
  -\frac{\Delta t}{\Delta x}\left[\tilde F_{i+1/2}-\tilde F_{i-1/2}\right],
\end{align}
where 
\begin{align*}
  \tilde F_{i+1/2} = \frac{1}{2}\sum_p|\hat\lambda_{i+1/2}^p|
  \left(1-\frac{\Delta t}{\Delta x}|\hat\lambda_{i+1/2}^p|\right)\tilde\W_{i+1/2}^p,
\end{align*}
and similarly for $\tilde F_{i-1/2}$, are the flux corrections with $\tilde\W_{i+1/2}^p$ 
being a TVD-limited version of $\W_{i+1/2}^p$. 
In this work, we consider the min-mod limiters, we refer to XX for more details. 

To understand the effect of choosing $s_{i+1/2}^p=\hat\lambda_{i+1/2}^p$ as the speed of $\W_{i+1/2}^p$, 
let us assume that no limiters are applied; i.e., $\tilde\W_{i\pm 1/2}^p\equiv \W_{i\pm 1/2}^p$.
By doing this, equation XX can be written as 
\begin{align*}
  u_i^{n+1}=u_i^n
  &-\frac{\Delta t}{2 \Delta x}
  \sum_p
  \left[\left(\hat\lambda_{i+1/2}^p -\frac{\Delta t}{\Delta x}|\hat\lambda_{i+1/2}^p|^2 \right)\W_{i+1/2}^p
  +
  \left(\hat\lambda_{i-1/2}^p +\frac{\Delta t}{\Delta x}|\hat\lambda_{i-1/2}^p|^2 \right)\W_{i-1/2}^p\right] \\
  &-\frac{\Delta t}{2\Delta x}
  \sum_p 
  \left[-\left(\lambda_{i+1/2}^{\max}-|\hat\lambda_{i+1/2}^p|\right)R_{i+1/2}\W_{i+1/2}^p
    +\left(\lambda_{i-1/2}^{\max}-|\hat\lambda_{i-1/2}^p|\right)R_{i-1/2}\W_{i-1/2}^p\right],
\end{align*}
where the first row corresponds to the standard Lax-Wendroff method written in terms of 
fluctuation-like quantities. 
Since $\lambda_{i\pm1/2}^{\max}\geq |\hat\lambda_{i\pm1/2}^p|$, 
the second row adds extra stabilization.
Note that if $R_{i\pm 1/2}=0$, we recover the standard second-order Lax-Wendroff method based on 
Roe's Riemann solver. 
In the numerical experiments in \S XX, we obtain second-order experimental 
order of convergence (EOC) for problems with smooth solution, which is expected since 
$R_{i\pm 1/2}\approx 0$ in smooth regions.
Also note that if $R_{i\pm 1/2}\approx 1$, which we expect to happen near steep gradients, 
we do not recover the Lax-Wendroff method based on Rusanov's Riemann solver. 
Instead, we obtain the first-order method XX with a flux correction. Assuming 
the CFL-like condition $\Delta t/\Delta x|\hat\lambda_{i+1/2}^p|-1\leq 0$, the flux correction is anti-diffusive, 
which slightly improves the accuracy of the first-order Rusanov's method. 
In our numerical experiments 
with smooth solutions, we obtain first-order EOC if we set $R_{i\pm1/2}=1$, $\forall i$. 
%
Therefore, the accuracy properties of method XX depend upon the behavior of $R_{i\pm 1/2}$. 
In smooth regions, where $R_{i\pm 1/2}\approx 0$, we recover second-order convergence;
near discontinuities, where $R_{i\pm 1/2}\approx 1$, we add extra dissipation that decreases the convergence 
to first-order. It is this extra dissipation that reduces the occurance of carbuncle instabilities 
in our numerical experiments. In smooth regions, the accuracy of the underlying Lax-Wendroff-LeVeque method 
based on Roe's Riemann solver is preserved.

\subsection{Entropy stable Riemann solver}

\clearpage

\section{Numerical results}
Description of the methods. We consider either the first-order method XX or the second-order Lax-Wendroff
type method XX. For each of these methods we use one of the following three Riemann solvers: 
\begin{itemize}
  \item LLF: Rusanov's or Local Lax Friederichs 2-wave Riemann solver reviewed in \S XX.
  \item Roe: standard Roe's average based 3-wave Riemann solver reviewed in \S XX. 
  \item Roe-fix: Roe's solver with entropy fix based on XX. 
  \item EV:  entropy dissipative 3-wave Riemann solver described in \S XX.
\end{itemize}

\subsection{Dam break problem on a dry bed}

\subsection{Dam break problem on a wet bed}
We consider a one-dimensional dam break problem on a wet domain. 
We follow the setup in XX. 
The domain is $\Omega=(0,L)$ with $L=10$, the initial condition is given by 
$hu(x,0)=0$ and 
\begin{align*}
  h(x,0) = 
  \begin{cases}
    h_l & \mbox{ if } x\in(0,x_0], \\
    h_r & \mbox{ if } x\in(x_0,L),
  \end{cases}
\end{align*}
with $x_0=5$, $h_l=0.005$ and $h_r=0.001$. 
The exact solution, which can be found in XX and references therein, is given by 
\begin{align*}
  h(x,t) = 
  \begin{cases}
    h_l, \\
    \frac{4}{9g}\left(\sqrt{gh_l}-\frac{x-x_0}{2t}\right)^2, \\
    \frac{c_m^2}{g}, \\
    h_r, 
  \end{cases}
\quad 
  u(x,t) = 
  \begin{cases}
    0, &\mbox{ if } x\leq x_A(t), \\
    \frac{2}{3}\left(\sqrt{gh_l}+\frac{x-x_0}{t}\right), & \mbox{ if } x_A(t) < x\leq x_B(t), \\
    2(\sqrt{gh_l}-c_m), & \mbox{ if } x_B(t)<x\leq x_C(t), \\
    0, &\mbox{ if } x\leq x_C(t),
  \end{cases}
\end{align*}
where $x_A(t)=x_0-t\sqrt{gh_l}$, $x_B(t)=x_0+t\left(2\sqrt{gh_l}-3c_m\right)$, 
$x_C(t)=x_0+t\frac{2c_m^2\left(\sqrt{gh_l}-c_m\right)}{c_m^2-gh_r}$ and 
$c_m$ is the solution of 
$-8gh_rc_m^2\left(\sqrt{gh_l}-c_m\right)^2+\left(c_m^2-gh_r\right)^2\left(c_m^2+gh_r\right)=0$.
We use $g=1$ and solve the problem up to the final time $T=5$ considering the 2nd-order Lax-Wendroff 
type method XX with the LLF, Roe and EV Riemann solvers. The solution with different refinement levels 
and each Riemann solver is shown in Figure XX. In Table XX, we summarize the results of a convergence test. 
Since the solution is non-smooth, we expect no more than first order convergence rates. Note that the 
results with the entropy dissipative solver EV are comparable against the results using Roe's solver. 
That is, inducing entropy dissipation via the EV Riemann solver does not degrade the high-order accuracy 
properties of the Roe's solver. 
In contrast, the accuracy and convergence rates using the LLF solver are clearly degraded.  

\subsection{The circular hydraulic jump in one dimension}

\subsection{Steady outflow}
Let us consider a problem similar to that in the previous section, but with an outflow boundary condition. 
That is, we consider the one-dimensional domain $\Omega=(0.1,1)$, the initial condition is 
$h(x,t=0)=0.1$, $hu(x,t=0)=0$, the left boundary condition is $h(0.1,t)=0.3$, $hu(0.1,t)=0.75$, 
and the right boundary condition is set to outflow, see the details in XX. 
If the flow is supercritical ($|F|>1$), the exact solution converges to a 
steady state profile given by the solution of the ODE XX. 
Depending on the initial condition, shocks might develop before the steady state is reached. 
Nevertheless, the steady state profile is smooth. 
In Figure XX, we show the solution at different times based on the secnd-order method XX with 
the EV Riemann solver. Additionally, in Table XX we summarize the result of a convergence study 
based on the first- and second-order methods XX and XX, respectively. 
We consider the Riemann solver in \S XX with Ri=0, Ri=1 and Ri computed via XX. 


\subsection{The circular hydraulic jump in two dimensions}
We consider two regimes. In the first regime, the physical instabilities are mild. 
By increasing the strength of the jump and other parameters, we consider a second 
and more strongly unstable regime. In both cases, we generate the CHJ as described in 
Section XX; that is, we impose supersonic inflow boundary conditions for the jet and 
subsonic outflow boundary conditions. 

We consider two types of domain and two corresponding computational grids. 
The two types of domains are two-dimensional squares and circular domains. 
For both cases we use structured grids. See an example of these domains and grids in Figures XX and XX, 
respectively.

\subsection{Regime 1: midly unstable flow}

Show the results with Roe's and Rusanov's solver. 
Show the results with the three proposed methodologies. 

\subsection{Regime 2: strongly unstable flow}

Show the results with Roe's and Rusanov's solver. 
Show the results with the three proposed methodologies. 

\clearpage
\section{Conclusions}

\appendix
\section{Bow shock problem for the shallow water equations}

\section{Other numerical studies}

\bibliographystyle{apalike}
\bibliography{refs}

\end{document}

