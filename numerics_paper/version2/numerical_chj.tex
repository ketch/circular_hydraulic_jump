\documentclass[preprint, 11pt]{article}

\usepackage[T1]{fontenc}
\usepackage[utf8]{inputenc}
\usepackage[hmargin=0.9in, vmargin=1.25in]{geometry}
\usepackage[babel,german=quotes]{csquotes}
\usepackage{subfig}
\usepackage{graphicx}
\usepackage[colorlinks=true,citecolor=blue,urlcolor=blue]{hyperref}
\usepackage{caption}
\usepackage{amsmath}          % writing mathematical formulas
\usepackage{amsthm}           % writing mathematical theorems 
\usepackage{amssymb}          % writing mathematical symbols
\usepackage{bm, amsfonts}     % writing bold mathematical symbols
\usepackage{xcolor}
\usepackage{fixmath}
\usepackage{tikz}
\usepackage{bm}
\usepackage{makecell}
\usepackage{mathdots}
\usepackage{algpseudocode}
\usepackage{algorithm}

\newcommand{\W}{{\mathcal W}}
\newcommand{\A}{{\mathcal A}}

\title{Numerical simulation and entropy dissipative cure of the 
  carbuncle instability for the shallow water circular hydraulic jump}

\author{
    David I. Ketcheson \and
    Manuel Quezada de Luna
}
\begin{document}

\maketitle

\begin{abstract}

\end{abstract}


\section{Introduction}

\subsection{The circular hydraulic jump}

\begin{itemize}
  \item Describe the formation of the circular hydraulic jump, some properties, etc. 
  \item Close this section stating that for some flow regimes, the CHJ becmes unstable and its 
    shape deviates from a circular shape. For high viscosity flows polygonal shapes are observed 
    and for low viscosity flows the shape appears to be chaotic. 
  \item We can use section 1.1 in the original version of the paper.
\end{itemize}

\subsection{Numerical shock instabilities}

\begin{itemize}
  \item Describe the carbuncle instability starting with Euler and going to SWEs. 
  \item Mention that we observe the cc instability for the CHJ as well. 
  \item Describe that in this case, the situation is more complicated than in other problems since we 
    expect the correct solution to have physical instabilities triggered by round off error. 
  \item Based on existing literature we know that we can use highly dissipative and robust methods to 
    avoid the formation of carbuncle instabilities. However, in this case, these methods also tend to 
    dissipate other important features of the solution unless the mesh is highly refined. 
  \item We are interested in methods that prevent the formation of cc instability in the CHJ 
    but preserve other features and physical instabilities. 
  \item There are multiple ideas and methods proposed to fix the cc insitability for the Euler 
    and recently for the SWEs. 
  \item It is believed that the cc instability occurs due to lack or reduced viscosity in the shear contact waves. 
  \item Some methods also propose to improve the entropy stability properties of the method. In particular, to induce or guarantee that the entropy is dissipated at shocks. 
  \item Based on some of these ideas we start with Rusanov's method using the cell averages of the solution. 
    This is a first-order method that is entropy stable and positivity preserving. We enhance its robustness by 
    overestimating the wave speed when strong shear is present in the solution. The result is a robust solver 
    that do not develop cc instabilities. However, the method is highly dissipative and, as a result, 
    many other important features of the solution are dissipated. 
  \item We then blend this highly dissipative method with a FV method based on Roe's average. 
  \item After this we introduce the second order corrections from Law-Wendroff's method. 
\end{itemize}

\subsection{Our contribution}
\begin{itemize}
\item We present a new problem where the cc instability occurs. In our opinion, this is a more challenging 
  problem than the standard bow problem since we expect the solution to be non-steady. 
  We aim to preserve instabilities that are inherently present in the CHJ for certain regimes. And, 
  at the same time, we aim to prevent the cc instability. 
\item We explore the use of entropy dissipation and stability as cure of the cc instability for the SWEs. 
\item We propose specific methods to blend the highly dissipative and robust Rusanov's RP and the 
  accurate 3-wave Roe's solver for the SWEs. In particular we propose methods that induce dissipation of entropy
  near the shocks and other methods that guarantee entropy stability. 
\item We demonstrate that these schemes prevent the formation of cc instabilities not only 
  in the well known bow shock problem, but also in the CHJ while preserving other important features of 
  the solution. 
\end{itemize}


\section{The shallow water circular hydraulic jump}
The shallow water equations in two (horizontal) dimensions are
\begin{subequations} \label{eq:sw}
\begin{align}
    h_t + (hu)_x + (hv)_y & = 0 \\
    (hu)_t + \left(hu^2 + \frac{1}{2}gh^2\right)_x + (huv)_y & = 0 \\
    (hv)_t + (huv)_x + \left(hv^2 + \frac{1}{2}gh^2\right)_y & = 0.
\end{align}
\end{subequations}
These can be written in vector form as
\begin{align}
    q_t + f(q)_x + g(q)_y & = 0.
\end{align}

\subsection{Semi-analytical steady solution under rotational symmetry}

By assuming rotational symmetry in \eqref{eq:sw}, one obtains the
system (in one spatial variable)
\begin{subequations} \label{eq:rsw}
\begin{align}
    (rh)_t + (rhu)_r & = 0 \label{mass1} \\
    (rhu)_t + (rhu^2)_r + r \left(\frac{1}{2}gh^2\right)_r = 0. \label{mom1}
\end{align}
\end{subequations}
Steady-state solutions of \eqref{eq:rsw} satisfy
\begin{subequations}
\begin{align}
    rhu & = C \\
    h'(r) & = \frac{h}{\frac{g}{\beta^2} r^3 h^3 -r} = \frac{h}{r} \cdot \frac{F^2}{1-F^2} \label{eq:dh}
\end{align}
\end{subequations}
for some $C$ independent of $r$.  Here $F=|u|/\sqrt{gh}$ is the Froude number.
We see that two types of steady profiles exist, depending on whether the flow
is subcritical ($|F|<1$) or supercritical ($|F|>1$).  No smooth steady solution can
include both regimes, since the right hand side of \eqref{eq:dh} blows up when $F=1$.

The steady, rotationally-symmetric circular hydraulic jump involves supercritical
flow for $r<r_0$ and subcritical flow for $r>r_0$, where $r_0$ is the jump radius.
The jump itself takes the form of a stationary shock wave.  The Rankine-Hugoniot jump
conditions specify that for such a shock,
\begin{align} \label{eq:RH}
    h_+ - h_- & = \frac{-3h_- + \sqrt{h_-^2 + 8 h_- u_-^2/g}}{2} = \frac{3h_-}{2}\left(\sqrt{1+\frac{8}{9}(F_-^2-1)}-1\right),
\end{align}
where the subscripts $+, -$ denote states just inside or outside the jump radius, respectively.

A steady-state, rotationally symmetric solution can be given for an annular region as follows:

\begin{enumerate}
    \item Specify the depth and velocity at the inner boundary (near the jet) and outer boundary.
    \item Integrate \eqref{eq:dh} from both boundaries.
    \item Find a radius $r_0$ where the matching condition \eqref{eq:RH} is satisfied.
\end{enumerate}
Due to the nature of solutions of \eqref{eq:dh}, it can be shown that the required jump
radius $r_0$ always exists if the prescribed flow is supercritical at the inner boundary
and subcritical at the outer boundary.

Numerical tests suggest that the solution obtained in this way is a stable equilibrium
of the reduced one-dimensional shallow water equations \eqref{eq:rsw}.  This
solution provides a useful starting point for testing the stability of the CHJ
as a solution of the full two-dimensional shallow water equations \eqref{eq:sw}.

\subsection{Formation of the CHJ}

\section{Numerical methods}
We consider the implementation of the Lax-Wendroff method in Clawpack. 
For a one-dimensional problem, the method can be written as follows:
\begin{align*}
  U_i^{n+1}=U_i^n-\frac{\Delta t}{\Delta x}\left[\A\Delta Q_{i-1/2}\right]
\end{align*}
where XX. In this section we focus on the first order method; i.e., we consider 

XX 

and propose different Riemann solvers. Once the Riemann solvers are defined for the low-order method, 
the second order corrections are incorporated following the same steps for all cases. Therefore, we focus our 
attention to design of the Riemann solvers. 

\subsection{Roe's Riemann solver}

\subsection{Rusanov's Riemann solver}

\subsection{Blended Rusanov's and Roe's solver based on entropy residual}

\subsection{Blended Rusanov's and Roe's solver based on Tadmor's entropy stability condition}

\subsection{Entropy stable enhancement of Roe's Riemann solver}

\subsection{Other alternatives}

\section{Numerical results}

We consider two regimes. In the first regime, the physical instabilities are mild. 
By increasing the strength of the jump and other parameters, we consider a second 
and more strongly unstable regime. In both cases, we generate the CHJ as described in 
Section XX; that is, we impose supersonic inflow boundary conditions for the jet and 
subsonic outflow boundary conditions. 

We consider two types of domain and two corresponding computational grids. 
The two types of domains are two-dimensional squares and circular domains. 
For both cases we use structured grids. See an example of these domains and grids in Figures XX and XX, 
respectively.

\subsection{Regime 1: midly unstable flow}

Show the results with Roe's and Rusanov's solver. 
Show the results with the three proposed methodologies. 

\subsection{Regime 2: strongly unstable flow}

Show the results with Roe's and Rusanov's solver. 
Show the results with the three proposed methodologies. 

\section{Conclusions}

\appendix
\section{Bow shock problem for the shallow water equations}

\section{Other numerical studies}

\bibliographystyle{apalike}
\bibliography{refs}

\end{document}

