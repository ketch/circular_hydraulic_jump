\documentclass[english,11pt]{article}

\usepackage{enumitem}
\usepackage{cite}
\usepackage{bbm}
\usepackage{ulem}
\usepackage{amsthm}
\usepackage[latin9]{inputenc}
\usepackage{babel}
\usepackage[hmargin=0.9in, vmargin=0.75in]{geometry}

\usepackage[colorlinks=true,citecolor=blue,urlcolor=blue]{hyperref}
\usepackage[dvipsnames]{xcolor}

\usepackage{url}
\usepackage{bm}

\usepackage{amsmath}
\usepackage[numbers]{natbib}
%\let\cite=\citet
\newcommand{\bfx}{{\bf x}}

\makeatletter
\providecommand{\tabularnewline}{\\}
\makeatother

%%---------------------------------------------------%%
%%---------------------- TITLE ----------------------%%
%%---------------------------------------------------%%
\begin{document}
\title{Response to Reviewers}
\maketitle

\vspace{1.5cm}
\noindent
Michael Dumbser \\
Editor \\
International Journal for Numerical Methods in Fluids \\

\vspace{1.5cm}

\noindent
Manuscript \# FLD-21-0079
\vspace{1.5cm}

\noindent
Dear Prof. Michael Dumbser,

\vspace{0.25cm}
\bigskip
The manuscript entitled
\begin{center}
``Numerical simulation and entropy dissipative cure of the carbuncle instability for the shallow water circular hydraulic jump''
\end{center}
by David I. Ketcheson and Manuel Quezada de Luna
has been revised following the comments made by the referees.

\bigskip
Letters detailing how the comments of each reviewer have been
addressed are enclosed. To facilitate your perusal of the revised
manuscript, changes concerning the comments of the first and second
reviewers are highlighted in red and green, respectively.


\vspace{0.25cm}
\bigskip\noindent 
With our best regards,

\vspace{0.5cm}
\bigskip\noindent
David I. Ketcheson and Manuel Quezada de Luna

\newpage
%%---------------------------------------------------------%%
%%-------------------- FIRST REVIEWER -------------------- %%
%%---------------------------------------------------------%%
\section*{Revision of Manuscript \# FLD-21-0079 -- Reviewer $\#1$}
We thank the reviewer for the comments and suggestions.
We have incorporated all of them.  For your convenience,
the modifications related to the comments made by reviewer $\#1$ and $\#2$
appear in red and green respectively in the new version of the manuscript. 
Statements in red below are comments from the reviewer.
In this letter, the comments made by the reviewer refer to the original
manuscript and our responses refer to the new manuscript. 

\bigskip
{\color{red}
  \begin{itemize}
    \item[(1)]
      Firstly, it is not clear why blending the Rusanov and Roe methods (Section 3) would fix the carbuncle
      using entropy residual as the guiding principle. From literature, it is obvious that Roe-type methods suffer
      from carbuncle whereas Rusanov-type schemes do not. But why using the entropy residual somehow
      makes the hybrid of the two methods remove the carbuncle? Is entropy-stability a factor (Eq. 13)? If yes,
      why?
  \end{itemize}
}
\noindent
Thank you for raising this import point. 
We have revised the introduction to emphasize our motivation for using the entropy residual as blending
criteria between Roe's and Rusanov's solvers. 

As the reviewer remarks, Rusanov-type solvers do not suffer from carbuncle instabilities. Therefore,
our aim is to use Rusanov's solver near shocks (where the carbuncle instability might appear) 
while using the less-diffusive Roe's solver elsewhere.
The task is to find a blending criteria that can robustly and sharply locate shocks.
The entropy residual is a natural choice since it is expected to vanish away
from shocks and has proven to be a successful tool for this; see for instance
\cite{guermond2011,guermond2018}. 
%
Near shocks the entropy residual is non-zero, so our blended solver induces dissipation of entropy near shocks,
similarly to the exact solution. Moreover, if the blended Riemann solver is used with the first order method (7),
the solution is provably entropy stable, as discussed in Section 3.3.
Using entropy stability to avoid carbuncle formation is not a new idea. For example, in \cite{ismail2009,ismail2009v2},
the authors obtain carbuncle-free solvers for the Euler equations by either inducing entropy dissipation or guaranteeing entropy stability. 
To our knowledge, no work has been done exploring similar ideas for the shallow water equations. However, based on
the success in the aforementioned references, it is reasonable to expect that entropy stability is a good tool to
eliminate carbuncle formation in the shallow water equations. 
%


\bigskip
{\color{red}
  \begin{itemize}
  \item[(2)]
    Assuming there is a solid reason for the proposed fix in the paper. It is also unclear how the indicator
    ($\theta$, after Eq. 19b) works to actively switch between Rusanov to Roe and vice-versa. The authors
    mention that $\theta=0$ means Roe-flux whereas $\theta=1$ recovers the Rusanov flux. What is the guiding
    principle to select the 'appropriate' $\theta$ values? This is somewhat discussed in 3.1 in between Eq. 
    16-19b but it is not intuitively clear to me how the entropy residual is used as a guide to select $\theta$.
    Moreover, what does the value in between $0<\theta<1$ means?
  \end{itemize}
}
\noindent
We provide a precise definition of $\theta_i$ in (19) and (21). The guiding principle to select $\theta_i$
is based on the entropy residual. As discussed before, using the entropy residual to locate shocks has proven
its worth in many works. 
Having $0<\theta<1$ produces a solver that combines the dissipation of both solvers; i.e.,
for those cases, the blended solver is more dissipative than Roe's but not as much as Rusanov's. 
We have revised the text in Section 3.1 accordingly. 

\bigskip
{\color{red}
  \begin{itemize}
  \item[(3)]
    If we continue to assume that (2) above can also be justified, I strongly suggest that the one or two of
    the test cases to be further evaluated. Dumbser et al. (04) and Kitamura et al. (AIAA 09) showed that the
    instability may manifest itself for a certain 'disturbance' in the test cases either via initial point values
    disturbance or grid disturbances, even if the fix works nicely on the non-disturbed conditions. I strongly
    suggest to vary the mesh as done in Kitamura et al. for the cylinder case (at least 3 different meshes for
    cylinder) and to include the residual plot. This is because the residual plot for convergence can easily show
    potential instability. The contour plot solution may look nice but the residual plot may show that it is
    moving toward instability.
  \end{itemize}
}
\noindent
For this problem, we do not initialize the flow with the steady state solution; we start
with a uniform flow given by $h_0=1$, $u_0=5$ and $v_0=0$ (i.e., we do not start with a shock).
After a very short time, a shock forms near the cylinder and travels to its steady state position.
Once at steady state, the shock is not aligned with a grid interface. Instead, the 
shock is resolved within a few cells (see Figure 3c). 
Since we start from a condition far from the steady state, 
perturbing the initital condition or the grid has no effect on the eventual shock that forms. We modified the text in
Section 4.3 to explain this.  In addition, we added a plot of the residual; see equation (29) and 
Figure 4. 
The residual stagnates when we use the blended solver with the second-order method (6); however, it 
continues to decrease if the first-order scheme (7) is considered. This suggests that the residual stagnates 
not due to the Riemann solver but due to the second-order corrections. 

\bigskip
{\color{red}
  \begin{itemize}
  \item[(4)]
    The hydraulic jumps with random perturbations (Section 5.4.3). This is good start for introducing a
    'disturbance' but may not be sufficient to justify carbuncle-free fix. Please try out the following
    disturbance. Let $(u_L, h_L)$ to be the initial left and $(u_R, h_R)$ the right states of the jump. 
    Introduce the following disturbance at the last cell prior to the jump such that 
    $u=\delta { u_L} + (1-\delta)u_R, h=\delta {h_L} + (1-\delta)$ where $0\leq{\delta}\leq{1}$. 
    Please try a discrete increment such that $\delta = 0.0, 0.1, 0.2,...,0.9, 1.0$. 
    Details can be found in Kitamura et al. (AIAA 2007, 2009). Run the proposed range of $\delta$
    for at least one of the hydraulic jump cases.
  \end{itemize}
}
\noindent
We added the requested simulation (and cited the corresponding reference) 
for the first hydraulic jump case; see Figure 10. We obtain a similar result as before; that is, 
when the Roe solver is used with any value of $\delta$, we obtain the carbuncle instability. 
With the blended solver, we do not obtain carbuncles for any value of $\delta$.

\bigskip
{\color{red}
  \begin{itemize}
  \item[(5)]
    I am a bit concerned when the authors mention in the conclusion that the proposed fix may not be fully
    entropy-stable. This is somewhat counter-intuitive of using entropy control to remove the instability as
    proposed throughout the paper. This comes back to my first question. Why is your proposed fix herein
    removes the carbuncle problem?
  \end{itemize}
}
\noindent
We have revised the conclusions to make it clear that we do propose an entropy-stable carbuncle-free scheme;
namely, the first-order method (7).  It is only the second-order method (6) that is not entropy-stable.
The second-order scheme does incorporate the modification made to the first-order scheme, which is
presumably the reason that it does not exhibit carbuncles.
In the design of numerical methods, it is often the case that some compromise must be made between conflicting
objectives, and this is such a case.  In the construction of the second order
method, we have chosen to give up the strict entropy-stable property in order
to achieve greater accuracy without compromising the carbuncle-free property.



\begin{thebibliography}{99}
  \bibitem{guermond2011}
    JL. Guermond, R. Pasquetti, B Popov,
    Entropy viscosity method for nonlinear conservation laws.
    {\it J. Comput. Phys.}, {\bf 230} (2011) 4248--4267.

    \bibitem{guermond2018}
      JL. Guermond, M. Nazarov, B. Popov, I. Tomas,
      Second-order invariant domain preserving approximation of the Euler equations using convex limiting.
      {\it SIAM J. Sci. Comput.}, {\bf 40(5)} (2018) A3211--A3239.

    \bibitem{ismail2009}
      F. Ismail and P.L. Roe,
      Affordable, entropy-consistent Euler flux functions II: Entropy production at shocks.
      {\it J. Comput. Phys.}, {\bf 228(15)} (2009) 5410--5436.

    \bibitem{ismail2009v2}
      F. Ismail, P.L. Roe, and H. Nishikawa,
      A proposed cure to the carbuncle phenomenon.
      {\it In Computational Fluid Dynamics}, (2006), 149--154.
\end{thebibliography}

%\bibliographystyle{plain}
%\bibliography{refs}
%\clearpage
%\bibliographystyle{plain}
%#\bibliographystyle{abbrvnat}
%\bibliography{ref.bib}

%%----------------------------------------------------------%%
%%-------------------- SECOND REVIEWER -------------------- %%
%%----------------------------------------------------------%%
\newpage
\section*{Revision of Manuscript \# FLD-21-0079 -- Reviewer $\#2$}
We thank the reviewer for the comments and suggestions.
We have incorporated all of them.  For your convenience,
the modifications related to the comments made by reviewer $\#1$ and $\#2$
appear in red and green respectively in the new version of the manuscript.
Statements in green below are comments from the reviewer.
In this letter, the comments made by the reviewer refer to the original
manuscript and our responses refer to the new manuscript.

\bigskip
{\color{OliveGreen}
  \begin{itemize}
  \item[(1)]
    When dealing with strong transcritical (or transonic, in the case of Euler equations) shocks, 
    the slowly-moving shock anomaly may appear 
    (see the thesis from Zaide \url{https://www.danielzaide.com/PDF/zaide_thesis.pdf})
    This anomaly is related to the discrete shock structure and appears when having a non-linear Hugoniot locus. 
    A spike in the momentum is observed downstream of the shock wave. Some authors have related the carbuncle to 
    the discrete shock structure and the unphysical values obtained in the cell containing the shock 
    (See Xie et al. \url{https://www.sciencedirect.com/science/article/pii/S0021999117306447?via\%3Dihub},
    Zaide and Roe \url{https://arc.aiaa.org/doi/10.2514/6.2011-3686}, etc). I recommend the authors to mention 
    this in the introduction and examine if the novel blended solver suffers from this anomaly (see comments below). 
    The performance of carbuncle-free solvers based on entropy arguments for the reduction of the slowly-moving 
    shock anomaly has not been yet assessed (to my knowledge).
  \end{itemize}
}
\noindent
Thank you for bringing these references to our attention.
We modified the introduction as suggested. In addition, in the new manuscript we investigate 
whether our blended solver suffers from this problem. 
Based on our numercal experiments, the profile of the momentum across the shock is non-monotone.
However, the location of steady shocks remains fixed; i.e., our solver does not suffer from 
the slowly-moving anomaly. 

\bigskip
{\color{OliveGreen}
  \begin{itemize}
  \item[(2)]
    A numerical fix for the slowly-moving shock anomaly which avoids adding additional dissipation is proposed 
    by Zaide in \url{https://www.danielzaide.com/PDF/zaide_thesis.pdf}.
    The authors are recommended to include this reference in the introduction, 
    since it includes a very detailed review on numerical shockwave anomalies.
  \end{itemize}
}
\noindent
Thank you for bringing this reference to our attention.  It is indeed highly relevant; we have
added a citation in the introduction as well as in appropriate places later in the manuscript.

\bigskip
{\color{OliveGreen}
  \begin{itemize}
  \item[(3)]
    The proposed method is designed for the homogeneous SWE. 
    Please explain if this idea can be easily extended to the SWE with bed elevation, friction, etc. 
    Is there any limitation when dealing with source terms?
  \end{itemize}
}
\noindent
We are interested in considering source terms and guaranteeing positivity preservation but 
we believe that constitutes a new project and, as a result, we would prefer to leave that 
for future work. We have included a remark about this in the conclusions. 
      
\bigskip
{\color{OliveGreen}
  \begin{itemize}
  \item[(4)]
    There is a paper by Navas-Montilla and Murillo 
    \url{https://www.sciencedirect.com/science/article/pii/S0021999118307496?via\%3Dihub}
    where the carbuncle and the slowly-moving shock anomaly for the SWE with bed elevation source term 
    have been examined together and some numerical fixes are proposed. This seems to be the only work 
    where source terms are considered in the study of shock anomalies for the SWE. 
    Please mention it in the introduction.
  \end{itemize}
}
\noindent
Thank you for bringing to our attention this extremely relevant work.  We have added a reference in
the introduction and have also expanded and modified our discussion of later results taking
this work into account.

\bigskip
{\color{OliveGreen}
  \begin{itemize}
  \item[(5)]
    In test case 4.1, flow over dry bed is considered but the details of the wet-dry interface 
    capturing method are omitted. Can you please mention or cite the algorithm for the wet-dry treatment?
  \end{itemize}
}
\noindent
Although we use the term ``dry bed'' in the section title, we note clearly in the text that we
use a non-zero depth $h_r=10^{-15}$.  Although we do not have a proof of positivity preservation
for our method, we do not observe violations of positivity.  Thus we avoid the
need to deal with wet-dry interfaces.  Note that we use the first-order method
(7) in order to reduce the potential for the appearance of negative depth
values.  This was already stated in the original manuscript, but we have made a revision
that emphasizes this.

%This does raise the question of how well the solver is able to avoid negative values of $h$
%in general.
%The first-order method with Rusanov's solver is positivity preserving (under a CFL-like time step restriction). 
%Using the other Riemann solvers and/or the high-order corrections might introduce violations of positivity.
%To minimize the loss of positivity, we used the first-order scheme for this test problem. 
%Importantly, Roe's and the blended solvers might still introduce violations of positivity, but 
%we did not observe that in our experiments. 

\bigskip
{\color{OliveGreen}
  \begin{itemize}
  \item[(6)]
    The convergence rates in Table 1 are suboptimal (around 0.4 for Roe solver), 
    whereas those in Table 2 achieve 1-st order of accuracy. Is this suboptimality caused by 
    the wet-dry interface? Please add a comment in the text.
  \end{itemize}
}
\noindent
Due to the entropy glitch, Roe's solver does not converge to the correct solution with Roe solver,
hence the drop in rate of convergence.
The other solvers do converge to the correct solution, but at a slower rate than
in Table 2 because Table 1 is obtained using the first-order scheme. 
We revised the text in Section 4.1 to improve clarity. 

\bigskip
{\color{OliveGreen}
  \begin{itemize}
  \item[(7)]
    In test case 4.3, can you add a plot of the solution for hu along x (at y=50)? 
    This will allow to visualize the possible presence of unphysical intermediate shock values 
    (slowly-moving shock anomaly) in the traditional solvers and determine if the blended solver 
    is able to overcome this issue. 
    See Case 4.1.2 from \url{https://www.sciencedirect.com/science/article/pii/S0021999118307496?via\%3Dihub}
    and include a plot like figure 4, where the evolution in time of hu is depicted and the shedding of 
    oscillations observed.
  \end{itemize}
}
\noindent
We appreciate the reviewer's suggestion. 
We have included the requested plots using the Rusanov and the blended solvers.
%In both cases, we obtain the non-physical spike in the momentum, which is expected due to the non-linear Hugoniot locus
%and the presence of intermediate states through the numerical shock.
In \cite[Section 4.1.2]{navas2019improved}, it is remarked that spurious oscillations in the transverse direction
are introduced with Roe's solver. If HLL(S) is employed, these oscillations are damped but are still present.
The authors also conjecture that increasing dissipation of the shear waves would reduce this effect even further.
Rusanov's solver adds more dissipation to the shear waves and, as conjectured by the authors,
the spurious oscillations are eliminated; see the top-right panel of our newly added Figure 4.
With the blended solver there are still small oscillations. 

Importantly, with the Rusanov or the blended solvers, the position of the shock appears to remain constant
(once the steady state is achieved).
To further confirm this, we run the simulation up to $t=160$ and plot the location of the shock along $y=50$.
Note that after $t\approx 55$, the location of the shock remains constant.


\bigskip
{\color{OliveGreen}
  \begin{itemize}
  \item[(8)]
    I think the presence of the aforementioned unphysical values (spike in hu) are observed in Figure 8b (right) 
    and Figure 9 (right), as there is a ring of amplitude above (red color) for hu separating the supercritical 
    and subcritical regions. Can you include some comments about this?
  \end{itemize}
}
\noindent
We agree with the reviewer. We have included a remark about this in Section 5.4.2 and in the conclusions.
Our blended solver is a convex combination of Rusanov's and Roe's solvers,
both of which produce the spike. Consequently, the blended solver also produces the spike. 
We believe that, within our framework, the key ingredient to solve this problem would be to modify
Rusanov's fluxes to eliminate the spike following for example \cite[Section 5.1]{navas2019improved}.
Afterwards, we can blend the modified Rusanov's fluxes with Roe's fluxes.
We leave this for future research but make a corresponding remark in the conclusions. 

\bigskip
{\color{OliveGreen}
  \begin{itemize}
  \item[(9)]
    In figure 7c, can you also plot $hu$?
  \end{itemize}
}
\noindent
We included the requested plot. As expected, the non-physical spike is clear.
We also added a remark about this in Section 5.4.1.

\bigskip
{\color{OliveGreen}
  \begin{itemize}
  \item[(10)]
    In figure 7c, the label of the $Y$ axis is missing.
  \end{itemize}
}
\noindent
Fixed as requested.

\bigskip
{\color{OliveGreen}
  \begin{itemize}
  \item[(11)]
    In case 5.4.3, the authors mention that the observed instabilities have a physical base and are not 
    produced by the numerical method. When saying ``physical instabilities'', what do you mean? 
    The SWE is a 2D model which consider depth-averaged quantities. Note that the real physics may involve 
    a more developed hydraulic jump (which involves vertical velocities) which cannot be represented by a 
    pure discontinuity as done with the SWE. Please clarify in the text the meaning of ``physical instabilities''.
  \end{itemize}
}
\noindent
We have revised the discussion of this result to be more clear and to avoid
referring to the instability in the simulations as ``physical''.  We do refer by
way of comparison to a physical instability of the CHJ that has been known for
some time
(see e.g. [8, Figure 2]).  While we agree that the shallow water model omits
details that may be important to such an instability, we are not aware of any result that
would preclude a close connection between what is observed here and in experiments.
We believe this is an interesting result that merits further research.

\bibliographystyle{plain}
\bibliography{refs}

\end{document}



