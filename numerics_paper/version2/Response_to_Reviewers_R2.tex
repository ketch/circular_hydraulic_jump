\documentclass[english,11pt]{article}

\usepackage{enumitem}
\usepackage{cite}
\usepackage{bbm}
\usepackage{ulem}
\usepackage{amsthm}
\usepackage[latin9]{inputenc}
\usepackage{babel}
\usepackage[hmargin=0.9in, vmargin=0.75in]{geometry}

\usepackage[colorlinks=true,citecolor=blue,urlcolor=blue]{hyperref}
\usepackage[dvipsnames]{xcolor}

\usepackage{url}
\usepackage{bm}

\usepackage{amsmath}
\usepackage[numbers]{natbib}
%\let\cite=\citet
\newcommand{\bfx}{{\bf x}}

\makeatletter
\providecommand{\tabularnewline}{\\}
\makeatother

%%---------------------------------------------------%%
%%---------------------- TITLE ----------------------%%
%%---------------------------------------------------%%
\begin{document}
\title{Response to Reviewers}
\maketitle

\vspace{1.5cm}
\noindent
Michael Dumbser \\
Editor \\
International Journal for Numerical Methods in Fluids \\

\vspace{1.5cm}

\noindent
Manuscript \# FLD-21-0079.R1
\vspace{1.5cm}

\noindent
Dear Prof. Michael Dumbser,

\vspace{0.25cm}
\bigskip
The manuscript entitled
\begin{center}
``Numerical simulation and entropy dissipative cure of the carbuncle instability for the shallow water circular hydraulic jump''
\end{center}
by David I. Ketcheson and Manuel Quezada de Luna
has been revised following the comments made by the referees.

\bigskip
Letters detailing how the comments of each reviewer have been
addressed are enclosed. To facilitate your perusal of the revised
manuscript, changes concerning the comments of the first and second
reviewers are highlighted in red and green, respectively.


\vspace{0.25cm}
\bigskip\noindent 
With our best regards,

\vspace{0.5cm}
\bigskip\noindent
David I. Ketcheson and Manuel Quezada de Luna

\newpage
%%---------------------------------------------------------%%
%%-------------------- FIRST REVIEWER -------------------- %%
%%---------------------------------------------------------%%
\section*{Revision of Manuscript \# FLD-21-0079.R1 -- Reviewer $\#1$}
We thank the reviewer for the comments and suggestions.
We have incorporated all of them.  For your convenience,
the modifications related to the comments made by reviewer $\#1$ and $\#2$
appear in red and green respectively in the new version of the manuscript. 
Statements in red below are comments from the reviewer.
In this letter, the comments made by the reviewer refer to the original
manuscript and our responses refer to the new manuscript. 

{\color{red}
  The authors have closely applied the reviewer's suggestions 
  and the quality of the paper has been improved. This manuscript contains 
  a highly valuable work, very relevant to the journal, which may spur 
  some new ideas in the field of numerical shockwave anomalies. 
  Therefore, I do recommend this paper for publication after a minor correction (see below).

  Before publication, I would ask the authors to remove the sentence in lines 14-15. 
  As far as I understand, the slowly moving shock anomaly does not necessarily 
  involve the oscillation of a steady shock. It also involves the presence of the spike. 
  When the shock is not steady and moves slowly, the spike swings and sheds oscillations 
  downstream. However, when the shock is steady, the spike may remain fixed.
}

\noindent
Following the reviewer's suggestion, we removed the sentence and now simply state that
Rusanov's solver does not perturb the location of steady shocks. 


%%----------------------------------------------------------%%
%%-------------------- SECOND REVIEWER -------------------- %%
%%----------------------------------------------------------%%
\newpage
\section*{Revision of Manuscript \# FLD-21-0079.R1 -- Reviewer $\#2$}
We thank the reviewer for the comments and suggestions.
We have incorporated all of them.  For your convenience,
the modifications related to the comments made by reviewer $\#1$ and $\#2$
appear in red and green respectively in the new version of the manuscript.
Statements in green below are comments from the reviewer.
In this letter, the comments made by the reviewer refer to the original
manuscript and our responses refer to the new manuscript.

\bigskip
{\color{OliveGreen}
  \begin{itemize}
  \item[(1)]
    To strengthen the notion of using entropy-control as a means to remove the carbuncle, 
    I strongly suggest to cite the most recent work in removing shock instability using 
    entropy control. Chizari et al.
    (CAMWA 2021) developed a more general (and more multidimensional) entropy-stable finite volume
    scheme which has many benefits including minimal sensitivity to grid skewness and does not 
    suffer from shock-instability as shown for the Quirk's problem (Section 5.3). 
    Perhaps, citation of this paper can be done in page 2 in the authors' paper where 
    references [19,20] are cited.

    Cell-vertex entropy-stable finite volume methods for the system of Euler equations 
    on unstructured grids, CAMWA 2021, Vol 98, pp 261-279.
  \end{itemize}
}

\noindent
Thank you for bringing this reference to our attention. We have included the reference as suggested. 

\bigskip
{\color{OliveGreen}
  \begin{itemize}
  \item[(2)]
    For the flow past a cylinder problem, I still believe that the choice grids will be a 
    crucial factor in fully understanding the instability unless one can develop a scheme 
    which is minimally sensitive to grid changes as done by Chizari et al (CAMWA 2021). 
    As shown in that paper, the classic entropy-stable FV scheme will eventually succumb to 
    shock instability if the grids are 'bad' enough but the newly developed entropystable 
    FV method (residual-distribution based) therein does not, since the latter is proven to be 
    almost non-affected by grid degradations. 

    In the authors' case. Even on a nice grid like the one used in 4.3, there are some 
    issues with convergence (Figure 4b) using the blended 2nd order scheme where there 
    seem to be limit cycle in the residual convergence. The limit cycle may be an onset 
    for instability if the simulation is ran long enough. This is the classic instability 
    issue that has troubled the carbuncle community for decades. The authors' claim
    that this is due to the second order influence, which is fair enough. 

    \begin{itemize}
    \item[(a)]
      However, how do you guarantee that the residual of the second order method will completely decay?
      Fig 4b does not indicate that this will happen. In other words, how do you know if the 
      second-order residual limit cycle will not manifest into shock instability if you ran the 
      simulations longer?      
    \end{itemize}
  \end{itemize}
}
\noindent
For a given mesh, we believe the residual will not decay anymore. 
This is a common problem with many 2nd order schemes that make use of limiters 
based on a posteriori fixes; see for instance \cite{kuzmin2020monolithic}. As stated in the 
paper, we believe the reason for the lack of convergence to a steady state is related the second-order 
correction (including the limiters). This is suggested by the fact that the blended solver with 
the first-order scheme does not prevent convergence to a steady state and since it is well-known that 
some a posteriori limiters inhibit convergence to steady state solutions.

We have run the simulation for a much longer time ($t=16000$ instead of $t=160$) and we still do not 
observe a shock instability. Of course this is not a guarantee that a problem will never occur. 
We have modified the text to discuss this problem. 

{\color{OliveGreen}
  \begin{itemize}
    \item[]
    \begin{itemize}
      \item[(b)]
        Looking at the magnitude of residual of $\mathcal{O}(10^{-3})$ for the blended 
        2nd order scheme. This is of the same magnitude with the residuals of the HLLEMCC 
        (Fig 4b). The authors state in page 14 that the HLLEMCC residual stagnates. 
        Is this is a bad thing? If yes, how would it be any different from the similar 
        residual behavior for the second order blended method?
    \end{itemize}
  \end{itemize}
}
\noindent
In the paper we state ``With HLLEMCC or the blended solver, the residual stagnates.'' That is, 
we acknowledge that with both solvers the residual stagnates. In our opinion, the fact that the 
residual stagnates is not necessarily a bad thing, assuming this does not lead to a 
serious shock instability, as remarked by the reviewer. We only have numerical evidence that 
this does not occur. In any case, having a method for which the residual of a steady state solution 
decays to zero (up to machine precision) is desirable. We have started a new line of research to 
fix this problem; however, we believe this new research should be included in a different publication. 
We have modified the text (in Section 4.3 and in the conclusions) to discuss this problem and 
to comment on the new line of work we are pursuing.

{\color{OliveGreen}
  \begin{itemize}
    \item[]
    \begin{itemize}
      Is the blended solution in Fig 3c based on the first or second order method? 
      I suggest to included the first order (Fig 3c?) and second order (Fig 3d?) 
      blended shock profiles to demonstrate the difference between the two on a contour plot.
    \end{itemize}
  \end{itemize}
}
\noindent
The solution is based on the second-order method. As suggested by the reviewer, we have included 
a new figure using the first-order method. 

\bibliographystyle{plain}
\bibliography{refs}

\end{document}



